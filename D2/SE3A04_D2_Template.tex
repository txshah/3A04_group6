\documentclass[]{article}

% Imported Packages
%------------------------------------------------------------------------------
\usepackage{amssymb}
\usepackage{amstext}
\usepackage{amsthm}
\usepackage{amsmath}
\usepackage{enumerate}
\usepackage{fancyhdr}
\usepackage[margin=1in]{geometry}
\usepackage{graphicx}
\usepackage{array}
\graphicspath{ {./images/} }
%\usepackage{extarrows}
%\usepackage{setspace}
%------------------------------------------------------------------------------

% Header and Footer
%------------------------------------------------------------------------------
\pagestyle{plain}  
\renewcommand\headrulewidth{0.4pt}                                      
\renewcommand\footrulewidth{0.4pt}                                    
%------------------------------------------------------------------------------

% Title Details
%------------------------------------------------------------------------------
\title{Deliverable \#2}
\author{SE 3A04: Software Design II -- Large System Design}
\date{}                               
%------------------------------------------------------------------------------

% Document
%------------------------------------------------------------------------------
\begin{document}

\maketitle	
\noindent{\bf Tutorial Number:} T03\\
{\bf Group Number:} G6 \\
{\bf Group Members:} 
\begin{itemize}
	\item Cass Braun
	\item Nehad Shikh Trab
	\item Savvy Liu
	\item Tvesha Shah
	\item Victor Yu
\end{itemize}

\section*{IMPORTANT NOTES}
\begin{itemize}
	%	\item You do \underline{NOT} need to provide a text explanation of each diagram; the diagram should speak for itself
	\item Please document any non-standard notations that you may have used
	\begin{itemize}
		\item \emph{Rule of Thumb}: if you feel there is any doubt surrounding the meaning of your notations, document them
	\end{itemize}
	\item Some diagrams may be difficult to fit into one page
	\begin{itemize}
		\item Ensure that the text is readable when printed, or when viewed at 100\% on a regular laptop-sized screen.
		\item If you need to break a diagram onto multiple pages, please adopt a system of doing so and thoroughly explain how it can be reconnected from one page to the next; if you are unsure about this, please ask about it
	\end{itemize}
	\item Please submit the latest version of Deliverable 1 with Deliverable 2
	\begin{itemize}
		\item Indicate any changes you made.
	\end{itemize}
	\item If you do \underline{NOT} have a Division of Labour sheet, your deliverable will \underline{NOT} be marked
\end{itemize}

\newpage
\section{Introduction}
\label{sec:introduction}
% Begin Section

This section should provide an brief overview of the entire document.

\subsection{Purpose}
\label{sub:purpose}
% Begin SubSection
State the purpose and intended audience for the document.
% End SubSection

\subsection{System Description}
\label{sub:system_description}
% Begin SubSection
Give a brief description of the system. This could be a paragraph or two to give some context to this document.

% End SubSection

\subsection{Overview}
\label{sub:overview}
% Begin SubSection
Describe what the rest of the document contains and explain how the document is organised (e.g. "In Section 2 we discuss...in Section 3...").

% End SubSection

% End Section

\section{Analysis Class Diagram}
\label{sec:analysis_class_diagram}
% Begin Section
This section should provide an analysis class diagram for your application.
% End Section


\section{Architectural Design}
\label{sec:architectural_design}
% Begin Section
This section should provide an overview of the overall architectural design of your application. Your overall architecture should show the division of the system into subsystems with high cohesion and low coupling.

\subsection{System Architecture}
\label{sub:system_architecture}
% Begin SubSection
The Gaim app utilizes a Data-Centric Architecture, specifically the blackboard architecture style for majority of the application. The agents are the independent experts (knowledge sources) that our user can interact. Synchronously, the blackboard, the active data store component, will take this information in, dividing the solution space and providing a non-deterministic answer and controlling the logic of the application.
\\
\begin{table}[h]
Within our system, the components are defined as the following:  
    \centering
    \renewcommand{\arraystretch}{1.3}
    \begin{tabular}{| m{5cm} | m{6cm} | m{4cm} |}
        \hline
        \textbf{Subsystem} & \textbf{Purpose} & \textbf{Architectural Style} \\
        \hline
        Classification Management & Start a search, submit an image, submit text, fill a survey & Blackboard \\
        \hline
        Generate a Report & View result, generate report, view score, save & Blackboard \\
        \hline
        Account Management & Create an account and log in to an account & Repository \\
        \hline
    \end{tabular}
    % \caption{Subsystems and Their Architectural Styles}
    % \label{tab:subsystems}
\end{table}

System relationships are defined in section 3.2 of this document. 
\\

In addition, three databases are present on the model level of this architecture. An account database for account details, a classification database with background information regarding species classification, and a search/report history database to track user saved searches. 
\\

Our system architecture incorporates the Repository and Blackboard architecture styles. The blackboard style is chosen because it supports multiple independent knowledge sources that can be independently called and allows for a logical data store to present a final output. This is beneficial as our app involves multiple experts/sources of input that the user can utilize, either exclusively or in tandem, to classify a species. Due to this functionality, Blackboard architecture is ideal for classification management as we can narrow down the solution space based on agents and manage them based on the status of data store logic. Additionally, in terms of further expansion, it will become very easy to incorporate additional experts/knowledge sources, giving the user more potential use cases for the application. Further, for the generated report subsystem, we can extend that functionality by creating agents for “report types” (rather than just one general agent) and have our data store update and logically control the information those report agents display. Moreover, the data and solutions we are presenting to our clients are non-deterministic, making blackboard architecture the ideal choice to ensure we are not relying solely on agent logic and have our data store making active decisions based on a set of facts. 
\\

Another one of the architectures we chose is the Repository Architecture Style, because it supports direct fetching of deterministic outputs from agents. This architecture style allows for large complex information systems where different components need to access different aspects of information. This is the type of situation the account management subsystem will require, as many users should be able to use the application at the same time. Furthermore, this style allows the account management system to easily access stored user information, making it easy for agents to create new accounts and full credentials for verification. This style supports data integrity, for backups and restores, which is ideal for account security. Overall, this architect style is ideal for an expanding new application, making it easy for us to manage user data, ensure data integrity and potentially expand our new user base. 

% End SubSection
\subsection{Subsystems}
\label{sub:subsystems}
% Begin SubSection
 Provide a list of your subsystems, with a brief description of each. Be sure to document its purpose and relationship to other subsystems.

% End SubSection

% End Section
	
\section{Class Responsibility Collaboration (CRC) Cards}
\label{sec:class_responsibility_collaboration_crc_cards}
% Begin Section
This section should contain all of your CRC cards.

\begin{itemize}
	\item Provide a CRC Card for each identified class
	\item Please use the format outlined in tutorial, i.e., 
	\begin{table}[ht]
		\centering
		\begin{tabular}{|p{7cm}|p{7cm}|}
		\hline 
		 \multicolumn{2}{|l|}{\textbf{Class Name:}} \\
		\hline
		\textbf{Responsibility:} & \textbf{Collaborators:} \\
		\hline
		\vspace{1in} & \\
		\hline
		\end{tabular}
	\end{table}

	\begin{table}[ht]
		\centering
		\begin{tabular}{|p{7cm}|p{7cm}|}
		\hline 
		 \multicolumn{2}{|l|}{\textbf{Class Name: (Specify Class Name)}} \\
		\hline
		\textbf{Responsibility:} & \textbf{Collaborator:} \\
		\hline
		Knows Start an image search & Start an image search \\
		Knows Start a textual search & Start a textual search \\
		Knows Start a survey search & Start a survey search \\
		Knows Photo Scanner & Photo Scanner \\
		Knows Error/timeout Message & Error/timeout Message \\
		Knows Current Status & Current Status \\
		Knows Search History & Search History \\
		Knows Success message & Success message \\
		Knows Search details (report) & Search details (report) \\
		Knows Account Management & Account Management \\
		\hline
		\end{tabular}
	\end{table}
	

	\begin{table}[ht]
		\centering
		\begin{tabular}{|p{7cm}|p{7cm}|}
		\hline 
		 \multicolumn{2}{|l|}{\textbf{Class Name: Error/Timeout Message (Boundary)}} \\
		\hline
		\textbf{Responsibility:} & \textbf{Collaborators:} \\
		\hline
		Handles time-out events & Classification Management Data Store \\
		Handles Image Submission error events & \\
		Handles Text submission error events & \\
		Handles Survey submission error events & \\
		Handles report generation error events & \\
		\hline
		\end{tabular}
	\end{table}
	
	\begin{table}[ht]
		\centering
		\begin{tabular}{|p{7cm}|p{7cm}|}
		\hline 
		 \multicolumn{2}{|l|}{\textbf{Class Name: Current Status (Boundary)}} \\
		\hline
		\textbf{Responsibility:} & \textbf{Collaborators:} \\
		\hline
		Knows Classification Management Data Store & Classification Management Data Store \\
		Knows User Search Status & \\
		\hline
		\end{tabular}
	\end{table}
	
	
\end{itemize}
% End Section

\appendix
\section{Division of Labour}
\label{sec:division_of_labour}
% Begin Section
Include a Division of Labour sheet which indicates the contributions of each team member. This sheet must be signed by all team members.
\newline
\newline
\textbf{Cass Braun}
\begin{itemize}
    \setlength\itemindent{2em}
\item Add contributions 
\end{itemize}
\includegraphics[width=0.3\textwidth]{Cass.jpg}

\textbf{Nehad Shikh Trab}
\begin{itemize}
    \setlength\itemindent{2em}
\item Add contributions 
\end{itemize}
\includegraphics[width=0.6\textwidth]{Nehad.png}

\textbf{Savvy Liu}
\begin{itemize}
\setlength\itemindent{2em}
\item Add contributions 
\end{itemize} 
\includegraphics[width=0.6\textwidth]{Savvy.png}

\textbf{Tvesha Shah}
\begin{itemize}
    \setlength\itemindent{2em}
    \item Section 3 -  Identify and explain the overall architecture of your system  
    \item Section 3 - Provide the reasoning and justification of the choice of architecture 
    \item Section 4 - CRC card Blackboard DataStore
    \item Section 4 - CRC card Error/Timeout Message (Boundary)
    \item Section 4 - CRC card Current Status (Boundary)
    \item Managed github set up and assisted with formatting
\end{itemize} 
\includegraphics[width=0.6\textwidth]{Tvesha.png}

\textbf{Victor Yu}
\begin{itemize}
    \setlength\itemindent{2em}
\item Add contributions 
\end{itemize}
\includegraphics[width=0.3\textwidth]{Victor.png}
% End Section

\end{document}
%------------------------------------------------------------------------------